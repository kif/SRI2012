Title: Fast azimuthal Integration

Authors: 	Jérôme Kieffer
			Dimitrios Karkoulis
			Jonathan P. Wright

PyFAI [1,2] is an open-source python library for Fast Azimuthal Integration
which provides 1D- and 2D-azimuthal regrouping with a clean programming interface and
tools for calibration.

The Python programming language was chosen for many  scientific application and
especially in data analysis where new algorithms can be tested thanks to the
numpy toolbox and the ipython interface  which represents an
alternative to Matlab (Figure 1). 
Python allows, in addition, programs to scale to large applications,
like PyMca or MxCube.
Missing building blocks for X-ray science have been developped inhouse like
FabIO [3] for image access, pyHST for tomographic reconstructions and
pyFAI for azimuthal regrouping which is presented here.

Fast integrations are obtained by the combination of an algorithm ensuring that
each pixel from the detector provides a direct contribution to the final
diffraction pattern and a parallel algorithm suitable for modern hardware.

Azimuthal integration

Azimuthal integration is a transformation from cartesian space to polar space
where the radial dimention is either a distance (radius), a momentum transfer
(q) or a scattering angle (2θ).
During this transformation, both local densities and total intensities are
conserved in order to obtain accurate quantitative results. 

Regrouping is performed using a histogram-like algorithm: 
Each pixel of the image is associated to its polar coordinates (r, χ) then a
pair of histograms versus r are built, one non weighted for measuring the
number of pixels falling in each bin and another weighted by pixel intensities;
after dark-current subtraction, and corrections for flat-field, solidangle and polarization.
The diffraction pattern is obtained by dividing the weighted histogram
by the number of pixels per bin. 
2D regrouping is obtained in the same way using two-dimensional histograms over
radial and azimuthal dimentions.
To avoid high frequency noise where pixel statistics are low; pixels are
splitted over multiple bins according to their spatial extension. 

While histograms are efficient on a single processor, they cannot efficiently be
run in parallel due to write access conflicts. This issue was addressed by
calculating and storing the full table of correspondance (or look-up
table) between any image pixels and the histogram bins. Different output bins
can then be processed in parallel making azimuthal integraton efficient on
multi-core systems and on graphics cards.

Performances

The performance of the code is monitored on a variety of computing hardware and
for different image sizes. 
Figure 2 shows the number of frames processed per second versus image size; 
larger numbers are better.
The maximum data throughput is of the order 200 Mpixel/second (800MB/s of float
data, outperforming the speed of most current hard disks),
and therefore depends on the computer having a sustained supply of data to be able to continue working at this speed.
For a typical image of 2048x2048 pixels, after loading and setting up, pyFAI
takes about 20 milliseconds per new frame to compute the 1D integrated profile
but sustained rate on integrating large datasets (thousands of images) is only
of 100 milliseconds per image due to reading bottleneck.  
 
Deployment

PyFAI has been used online on a couple of SAXS beamlines at ESRF (Duble - BM26
and BioSaxs - BM29) integrated into python servers where it can process images
at 1 Hz; limited by the network and disk latencies.
To overcome this limitation, pyFAI is being integrated into the image
acquisition server LImA, providing integrated images directly as output of the
camera. In this configuration, performances were measured at 30 Hz.
Many beamlines showed great interest for those developments among them ID02,
ID11, ID13 and ID22. 

Status

The PyFAI code is fully open source under the GPL license and available from http://github.com/kif/pyFAI. 
PyFAI is packaged and available in common Linux distributions like Debian 7.0 and Ubuntu 12.04. 
Installer packages for Windows and MacOSX are available on https://forge.epn-campus.eu/projects/azimuthal/files. 
The software depends on : python 2.6+, NumPy, SciPy, matplotlib and FabIO. 
Thanks to image input code from the FabIO library, the code is already compatible with at least 20 detectors from 12 different manufacturers. 

Conclusions 

PyFAI is a novel library for azimuthal integration which provides
geometric equivalence with other programs and offers a clean programming
interface.

PyFAI is the first implementation of an azimuthal integration algorithm on a
graphic card as far as we are aware of, and exhibits a twenty-fold
speed up compated to existing implementations.

Acknowledgments

The authors would like to express their most sincere appreciation to their
colleagues and especially M. Sánchez del Río for suggesting
the usage of weighted histograms; P. Bösecke for the provision of the
experiment geometry setup; M. Burghammer for precise specifications; V. A.
Solé for his expertise on developing native code under Windows, F.-E. Picca
for the documentation and S. Petitdemange for helping with the integration into
LImA.
Porting pyFAI to \textsc{gpu} would have not been possible without the financial
support of LinkSCEEM-2 (RI-261600).

References:
[1] Jérôme Kieffer,
			Dimitrios Karkoulis,
			Journal: Journal of Physics: Conference Series (2013)
			vol: SRI2012
			Accepted
[2] 	Jérôme Kieffer,
				Jonathan P. Wright,
				Journal: Powder Diffraction 
				Submitted (Expected July 2013)				
[3] Erik B. Knudsen,
		Henning O. S{\o}rensen,
    	Jonathan P. Wright,
    	Gaël Goret,
    	Jérôme Kieffer;
    	Journal: Journal of Applied Crystallography (2013),
    	Accepted

Figure 1: Example of interactive use of FabIO and pyFAI in the notebook edition
of IPython.

Figure 2: Performances (frames processed per second) measured on a beamline
(ID13) workstation versus image size (in mega-pixel):
